\section{Serial Monitor}
Monitor serial adalah jendela yang menunjukkan data yang dipertukarkan antara arduino dan komputer selama operasi, sehingga Anda dapat menggunakan monitor seri ini untuk menampilkan nilai hasil operasi atau pesan debug. Selain melihat data, Anda juga dapat mengirim data ke Arduino melalui seri monitor ini, dengan memasukkan data dalam kotak teks dan menekan tombol kirim untuk mengirim data. Yang penting untuk diperhatikan adalah menyamakan baudrate antara seri monitor dan papan Arduino. 
Untuk menggunakan kemampuan komunikasi serial ini, di Arduino, di bagian kosong dari pengaturan (), ini dimulai dengan instruksi Serial. Diikuti oleh nilai baudrate.
\Penulisan objek  dan fungsi  pada library arduino adalah  dan namaobjek dan nama fungsi
contoh :  Serial.read()  , artinya  kita memanggil fungsi read() dari  objek bernama Serial .
Data yg dikirim  ke serial port  akan dikirim ke buffer pengirim (Tx buffer)  begitupun data yg diterima  adalah data yg diambil  dari  buffer penerima (RX buffer).

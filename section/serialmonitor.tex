\section{Serial Monitor}
Berikan informasi tentang komputer dan informasi komputer Anda tanpa akses ke informasi apa pun terkait penggunaan sistem operasi sistem operasi komputer Arduino. Arduino adalah pengontrol mikro single-board open-source, yang merupakan salah satu platform dengan koneksi kabel, yang akan memungkinkan Anda untuk memasang listrik. AVR Atmel dan perangkat perekaman data yang paling efektif untuk ekspor.
Monitor serial adalah jendela yang menunjukkan data yang dipertukarkan antara arduino dan komputer selama operasi, sehingga Anda dapat menggunakan monitor seri ini untuk menampilkan nilai hasil operasi atau pesan debug. Selain melihat data, Anda juga dapat mengirim data ke Arduino melalui seri monitor ini, dengan memasukkan data dalam kotak teks dan menekan tombol kirim untuk mengirim data. Yang penting untuk diperhatikan adalah menyamakan baudrate antara seri monitor dan papan Arduino. 
Untuk menggunakan kemampuan komunikasi serial ini, di Arduino, di bagian kosong dari pengaturan (), ini dimulai dengan instruksi Serial. Diikuti oleh nilai baudrate.
Menulis artikel dan fungsi di perpustakaan arduino adalah: namaobjek.namafungsi
Sebagai contoh, Serial.read (), yang berarti kita memanggil fungsi read () dari objek yang bernama Serial.
Data yang dikirim ke Cisco akan dikirim ke buffer Tx dan informasi yang diterima adalah informasi yang diterima dari penerima buffer (buffer RX).
Buffer adalah area penyimpanan memori ketika dipindahkan antara dua perangkat atau antara perangkat dan aplikasi. Buffering adalah untuk tiga alasan
\subsection Penulisan objek  dan fungsi  pada library arduino adalah  dan namaobjek dan nama fungsi
contoh :  Serial.read()  , artinya  kita memanggil fungsi read() dari  objek bernama Serial .
Data yg dikirim  ke serial port  akan dikirim ke buffer pengirim (Tx buffer)  begitupun data yg diterima  adalah data yg diambil  dari  buffer penerima (RX buffer). Sebuah data yang dikirm akan diterima oleh arduino  dalam bentuk ASCII. Misalnya  program arduino  mengirim huruf  A maka yang akan dikirim sebenarnya adalah 1 byte code ASCII yaitu 65 .  Jika kita mengirim angka 1,2,3 maka sebetulnya yg dikirm adalah 3 byte data ASCII yaitu 49, 48, dan 50 . Selengkapnya anda bisa lihat tabel ASCII berikut : 
Karena data yang diterima arduino juga dalam bentuk ASCII (char) maka kita tidak langsung melakukan operasi matematis data ASCII pada program sketsa arduino. Data ASCII dari buffer serial harus diubah terlebih dahulu ke nomor bentuk / nomor int. misalkan kita menerima data '4' dan '3' maka jika kita langsung menjumlahkan '3' + '4' = 103
\subsection(Fungsi-fungsi yg tersedia pada serial Arduino)
if (Serial): Untuk memeriksa apakah Port sudah siap
Serial.available (): Untuk memeriksa apakah data sudah ada di buffer penerima
Serial.begin (): untuk mengatur kecepatan transmisi data
serial.end (): Untuk menonaktifkan pin rx dan tx sebagai fungsi serial dan kembali sebagai pin I / O
Serial.find (): cari data string dalam buffer
Serial.findUntil (): berfungsi untuk mencari buffer data sampai panjang data / terminator yang ditentukan ditemukan
Serial.flush (): berfungsi untuk menunggu semua data terkirim
Serial.parseFloat (): berfungsi untuk mengambil data float pertama dari data dalam buffer serial.
serial.parseInt (): berfungsi untuk mengambil data bilangan bulat pertama dari data dalam buffer serial.Serial.peek (): mengambil data berikutnya di buffer penerima
Serial.print (): mengirim data ASCII
Serial.println (): senddata ASCII + CR, LF (masukkan kode)
Serial.read (): membaca data yang diterima
Serial.readBytes (): membaca data byte yang diterima
Serial.setTimeout (): menetapkan batas waktu maksimum untuk transmisi data.
Serial.write (): mengirim data byte (numerik)
Serial.serialEvent (): fungsi ini akan dipanggil ketika data tiba / diterima. Kerjanya seperti interupsi serial.

\subsection
Langkah-langkah untuk membuat program sketsa komunikasi serial
1. tetapkan baud rate misal 9600 dengan fungsi serial.begin (9600) dalam fungsi void setup (). Kecepatan yang tersedia termasuk 300, 1200, 2400, 4800, 9600, 14400, 19200, 28800,38400, 57600, 115200 (lihat dokumentasi masing-masing jenis arduino).
2. untuk menerima data cek apakah ada data di Rx Buffer dengan fungsi serial.avalable ()
jika data tersedia nilai kembalian = true jika data adalah nilai kembalian kosong = salah.
if (serial.available ()> 0)
3. mengambil data dari penerima buffer: serial.read (), return value adalah 1 byte data pertama di RX Buffer.
misalnya buffer RX byte pertama berisi char 2
kemudian char data = serial.read (); // data = '2' -> karakter
byte data = serial.read (); // data = 50 -> numerik
4. Untuk mengirim data dapat digunakan
serial.print (dataygdikirim): mengirim data ascii
serial.println (dataygdikirim), plus masukkan kode (CR dan LF)
serial.write (dataygdikirim): mengirim data byte